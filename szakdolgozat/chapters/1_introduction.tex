\Chapter{Bevezetés}

Az adatok elemzése és felhasználás nem új keletű dolog. Ha jobban belegondolunk mi magunk is adatokat elemezünk nap mint nap. Gondolok itt olyan dolgokra mint, amikor van egy megbeszélésünk 13:00-kor, mi ránézzünk az óránkra, látjuk hogy 12:15 van, valamint tudjuk, hogy az út az adott helyre 30 percet vesz igénybe. Ezekből az információkból két dolgot tudtunk meg. Az egyik, hogy legkésőbb 12:30-kor el kell indulnunk, hogy oda érjünk a megbeszélésünkre, valamint, hogy maximum 15 perc alatt indulásra késznek kell lennünk.

Azonban az ember mellett különböző cégek és vállalkozások is dolgoznak fel adatokat. Ezeknek a mennyisége és bonyolultsága viszont már nem olyan amit 1-1 ember könnyen értelmezni tud. Itt jön képbe az informatika ami, mint rengeteg más területre ide is betört. A 2020-as évre minden egyes nap 2,5 millió terrabyte adatot generáltunk. Ez lebontva körülbelül 1,7 MB adatot jelent emberenként minden egyes másodpercben kezdve a napi 3,5 milliárd google kereséstől egészen a 300 milliárd elküldött email-ig. Az olyan techóriásokról mint a Google, Facebook, Twitter, Amazon tudjuk, hogy ezeket az adatokat tökéletesen fel tudják használni arra, hogy személyre szabott reklámokat, posztokat mutassanak nekünk, hogy több időt töltsünk el a platformjaikon így nekik még több adatot, pénzt generáljunk.

Szakdolgozatomban ennek a tömérdek mennyiségű adatnak, amely különböző forrásokból jön létre csupán egy kis szeletét fogom elemezni a vendéglátásban, azonbelül is a szálláshelyek iparágában. Hotel foglalási adatok fogok elemezni gépi tanulás segítségével. Az adathalmaz elemzésének a célja egy olyan modell létrehozása ami a foglalás pillanatában meg tudja mondani egy foglalásról, hogy le lesz-e mondva vagy sem. Az adatok elemzéséből különböző érdekes dolgokat is meg lehet állapítani, amit szakdolgozatom későbbi részeiben be is fogok mutatni. A szállodaiparban egy ehhez hasonló betanított modell jól jöhet a szállodáknak, hiszen ha látnak egy olyan foglalást ami nagy valószínűséggel le lesz mondva, akkor megkockáztathatják, hogy elérhetővé teszik azt a szobát egy újabb foglalásnak, így próbálva maximalizálni a profitot. A szálloda kockázata pedig annál kisebb, minél jobb paraméterekkel van betanítva egy modell, hiszen annál pontosabban tudja megmondani egy foglalásról, hogy akik lefoglalták a szállást valóban meg fognak-e érkezni.

