\chapter{Összefoglalás}

Jó döntésnek gondolom, hogy a Python programozási nyelvet választottam alapul az adatok elemzéséhez és a gépi tanulási modellek megalkotásához, ugyanis mivel manapság mind a nyelv mind pedig a téma elég felkapott az internetet így rengeteg segédletet találtam online aminek a nagy többsége naprakésznek bizonyult. Ha mégegyszer neki kéne fognom a szakdolgozatom megírásának biztos vagyok benne, hogy ugyanígy a Python mellett döntenék. A munka folyamán különböző programkönyvtárakkal dolgoztam, ezeknek is tökéletes dokumentációja amik az egyes könyvtárak online felületein érhetőek el.

Számomra a szakdolgozatom során a legérdekesebb dolog az adathalmazzal kapcsolatban a korrelációs mátrix volt. Ugyanis ennél a résznél láthattuk pontosan, hogy melyek azok a változók amik leginkább befolyásolják a lemondást. Az számomra is gyanús volt, hogy hány nappal foglalták előtte a szállást mielőtt megérkeztek elég szoros köze lesz, hogy egy adott foglalást milyen valószínűséggel mondanak le, azonban azt sose gondoltam volna, hogy a különleges kérések szám is jelentősen befolyásolja a foglalásokat.

A való életben való felhasználását, a következőképpen tudnám elképzelni: A legjobban teljesító modellt egy asztali vagy webes alkalmazásab lehetne implementálni, majd amikor egy foglalás érkezik az adott szálláshelyre ez a program rögtön oda írna a foglalás mellé egy számot, ami azt mutatná meg, hogy milyen valószínűséggel lesz lemondva a foglalás. Az adatkészletet amin a modell tanul bővíteni lehetne a hotel saját foglalási adataival vagy ha elég adat van csak abból tanítani a modellt, hogy még pontosabban tudja előrejelezni a lemondást. Valamint mivel az emberek jó időben szeretnek nyaralni érdekes lehet, még összevetni a lemondási adatok az időjárással is. Az ötletemet egy példával szemléltetném. A szállást március 14.-én foglalták le március 21-25 közötti időszakra. Ekkor az időjárás előrejelzés napos és enyhén borús időjárást jósolt az adott időszakra. Azonban március 18.-án már esőt mondott több napra is és ezen a napon mondták le a foglalást. Van-e összefüggés az időjárás előrejelzés és a foglalási valamint lemondási kedv között?